\chapter{Описание подходов к решению задачи}

Данный раздел посвящен описанию походов и алгоритмов, позволяющих решить
поставленную задачу.

\section{Общие положения}

При решении поставленной задачи предполагается, что ссылка на рисунок и его
описание находятся в одном предложении. Таким образом, текстовый анализ
проводится отдельно для каждого предложения, поскольку предполагается, что
содержание одного предложения не зависит от содержания другого.

После параграфа должны быть расположены один или несколько рисунков, ссылки на
которые были найдены в предложениях. Для этого в ходе анализа предложения
производится поиск и загрузка изображений из Интернета. Вставка рисунков в
документ производится при помощи TeX-макросов.

Таким образом, в результате анализа отдельного параграфа происходит следующее:
\begin{itemize}
    \item происходит анализ и вставка ссылок для отдельных приложений;
    \item добавляются служебные параграфы, состоящие из TeX макросов;
    \item в указанный каталог загружаются найденные в Интернете изображения.
\end{itemize}

\section{Извлечение описания рисунка из предложения}

Прежде чем приступить к поиску и извлечению описания рисунка, необходимо
представить предложение в виде списка термов, для чего возможно использовать
инструменты естественной обработки языка.

Обработка предложения состоит из следующих этапов:
\begin{enumerate}
    \item токенизация (сегментация) по словам;
    \item приведение слов в начальную форму и определение частей речи;
    \item выделение термов.
\end{enumerate}

Этапы обработки предложения приведены в таблице \ref{tab:parse-example-1}.

\noindent
\begin{table}[!ht]%
\caption{Этапы обработки предложения}\label{tab:parse-example-1}%
\begin{tabularx}{\textwidth} {
    | >{\small\hsize=0.35\hsize\raggedright\arraybackslash}X
    | >{\small\hsize=0.65\hsize\raggedright\arraybackslash}X | }
\hline
Оригинальное предложение & Показанный на рисунке
гидродинамический удар, безусловно, подрывает смысл жизни.\\

\hline

Токенизация & {[}'Показанный', 'на',
'рисунке', 'гидродинамический', 'удар', ',', 'безусловно', ',', 'подрывает',
'смысл', 'жизни', '.'{]} \\

\hline

Приведение слов в нормальную форму и определение частей речи & {[}'Показать',
'на', 'рисунок', 'гидродинамический', 'удар', ',', 'безусловно', ',',
'подрывать', 'смысл', 'жизнь', '.'{]} \\

\hline

Выделение термов & {[}'рисунок', 'гидродинамический удар', 'смысл жизни'{]} \\
\hline
\end{tabularx}
\end{table}

Среди списка термов выделим два особенных:

\begin{itemize}
    \item терм-триггер - слово <<рисунок>>;
    \item терм-описание - искомое описание рисунка.
\end{itemize}

Терм-описание с большой вероятностью будет располагаться рядом с триггером.
Опираясь на этот факт, можно составить достаточно простой алгоритм для его
обнаружения.

Если терм $T_i$ - слово-триггер <<рисунок>>,то $T_{i+1}$ будет описанием этого
рисунка. Таким образом, в приведенном примере описанием рисунка является
словосочетание <<гидродинамический удар>>.

Однако, данный подход не учитывает случаи, когда терм-описание располагается
перед триггером. Рассмотрим следующий отрывок предложения: <<...бозе-конденсат,
изображенный на рисунке>>. В данном случае, термы будут расположены в следующем
порядке: <<{[} 'бозе-конденсат', 'рисунок' {]}>>.

Для обработки обоих случаев используем будем использовать следующее правило:
если первый терм предложения - триггер, то описанием будет второй терм
предложения; в ином случае описанием будет являться предшествующий триггеру
терм. Данный подход применим, если в предложении хотя бы два терма, и хотя бы
один из них является триггером.

Данное правило, представленное в псевдокоде, изображено на рисунке~%
\ref{fig:term-pseudocode}. Описание рисунка будет записано в переменную
\emph{context}, либо ее значение будет пустым, если в предложении
нет ссылок на рисунок.

\begin{codewrap}[0.85]
    \begin{minted}[fontsize=\small]{pascal}
string context = ""
if terms.length >= 2 and terms.contains(trigger) then
    context = terms[0] == trigger ? terms[1] : terms[i - 1]
end
    \end{minted}
    \caption{}\label{fig:term-pseudocode}%
\end{codewrap}

Полученное описание рисунка добавляется в список описаний всех рисунков в
тексте, где ему присваивается индекс, который будет использован для создания
ссылки и имени соответствующего файла изображения.

\section{Вставка ссылки на рисунок в текст}

Чтобы определить индекс символа в предложении, в которое должна быть внедрена
ссылка, можно использовать следующий алгоритм:

\begin{itemize}
    \item найти все совпадения с регулярным выражением \emph{(рисун(ок|ке|ка))};
    \item разбить предложение в точке, определенной окончанием этого совпадения;
    \item соединить полученные части предложения с ссылкой в порядке: \emph{
        часть1 + ссылка + часть2
    }
\end{itemize}

\section{Выводы по разделу}

В данном разделе были рассмотрены следующие аспекты анализа текста:
\begin{itemize}
    \item выбрана анализируемая единица текста - предложение;
    \item приведен алгоритм выделения ключевых термов;
    \item приведен алгоритм вставки ссылок на изображения;
\end{itemize}

В данном разделе были упомянуты, но не рассмотрены следующие алгоритмы,
необходимые для решения задачи:
\begin{itemize}
    \item токенизация;
    \item приведение слова в нормальную форму и определение части речи;
    \item составление термов.
\end{itemize}

Использование данных алгоритмов возможно, если задействовать библиотеки
обработки естественного языка для выбранного языка программирования.

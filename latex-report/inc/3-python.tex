\chapter{Реализация на языке Python}

В данном разделе разобран пример решения задачи на языке Python. Полный текст
программы приведен в приложении \ref{app:program}.

\section{Используемые библиотеки}
Для токенизации по предложениям используется библиотека \emph{nltk}.

Для выделения термов используется библиотека \emph{rutermextract}
\cite{rutermextract}. Она использует регулярные выражения для разбора по словам
и \emph{pymorphy2} в качестве морфологического анализатора. \emph{pymorphy2}
\cite{Korobov}, в свою очередь, использует морфологический корпус OpenCorpora
для определения части речи и нормальной формы слова.

На рисунке \ref{fig:import-code} показан импорт необходимых функций.

\begin{codewrap}[0.5]
\begin{minted}[fontsize=\footnotesize]{python}
from nltk import sent_tokenize
from rtm.rutermextract import TermExtractor
import re
from latex import escape
\end{minted}
\caption{}\label{fig:import-code}
\end{codewrap}

В данном случае используется \emph{sent_tokenize} для токенизации по
предложениям, модифицированная версия \emph{TermExtractor}, которая не
сортирует термы каким либо образом, стандартная библиотека регулярных выражений
\emph{re} и функция \emph{escape}, экранирующая названия вставляемых в макросы
изображений.

\section{Разбиение текста}

Код функций для преобразования текста на уровне параграфов показан на рисунке
\ref{fig:transform-par}.

\begin{codewrap}[1]
\begin{minted}[fontsize=\footnotesize]{python}
def handle_text(text):
    try:
        return "\n".join(transform_paragraph(p) for p in text.split("\n"))
    finally:
        close_downloader()


def transform_paragraph(p):
    image_pars = ""
    output_text = ""

    for sent in sent_tokenize(p, 'russian'):
        new_sent, latex = transform_sent(sent)
        output_text += new_sent + " "
        if latex:
            image_pars += "\n" + latex

    return output_text + image_pars
\end{minted}
\caption{}\label{fig:transform-par}
\end{codewrap}

Функция \emph{transform_paragraph} разбивает параграф на отдельные предложения и
обрабатывает каждое из них через \emph{transform_sent}, которая также возвращает
код макроса для вставки изображения. Полученные макросы впоследствии добавляются
в конец преобразованного параграфа.

\section{Анализ предложений}

На рисунке \ref{fig:transform-sent} показан код функции \emph{transform_sent},
анализирующей предложение и возвращающей его преобразованный вариант вместе с
кодом для вставки рисунка, если ссылка на него была обнаружена в предложении.

\begin{codewrap}[0.85]
\begin{minted}[fontsize=\footnotesize]{python}
def terms_contain_trigger(sent_terms):
    for term in sent_terms:
        if "рисунок" in term.normalized:
            return True
    return False


unique_idx = 0
term_extractor = TermExtractor()


def transform_sent(sent):
    global unique_idx
    sent_terms = term_extractor(sent)
    if len(sent_terms) < 2 or not terms_contain_trigger(sent_terms):
        return sent, None

    img_file = "image%s" % str(unique_idx)
    unique_idx += 1

    context = extract_context(sent_terms)

    tex_text = create_tex_image(context, img_file)
    download_image(context, img_file)

    img_ref = "fig:%s" % img_file
    return insert_ref(sent, img_ref), tex_text
\end{minted}
\caption{}\label{fig:transform-sent}
\end{codewrap}

Термы извлекаются при помощи экземпляра \emph{TermExtractor}, после чего
проводится извлечение описания рисунка с использованием \emph{extract_context},
код которой приведен на рисунке \ref{fig:extract-context}.

\begin{codewrap}[0.75]
\begin{minted}[fontsize=\footnotesize]{python}
def extract_context(sent_terms):
    if "рисунок" in sent_terms[0].normalized:
        return sent_terms[1].normalized

    for i in range(1, len(sent_terms)):
        if "рисунок" in sent_terms[i].normalized:
            return sent_terms[i - 1].normalized
\end{minted}
\caption{}\label{fig:extract-context}
\end{codewrap}

Для создания макроса вставки используется \emph{create_tex_image}, создающая
подпись вида <<Рисунок 1 - <context> >>. Ее код показан на рисунке
\ref{fig:create-tex-image}.

\begin{codewrap}[1]
\begin{minted}[fontsize=\footnotesize]{python}
def create_tex_image(context, img_file):
    return "\\insertimage{%s}{%s}" % (escape(img_file), context.capitalize())
\end{minted}
\caption{}\label{fig:create-tex-image}
\end{codewrap}

Каждому файлу присваивается уникальное имя вида \emph{image<index>}. Затем
производится поиск и сохранение файла изображения с указанным именем с
использованием \emph{download_image}. Последним этапом является вставка ссылок
на рисунки при помощи функции \emph{insert_ref}, код которой приведен на рисунке
\ref{fig:insert-ref}.

\begin{codewrap}[0.75]
\begin{minted}[fontsize=\footnotesize]{python}
def insert_ref(sent, img_ref):
    match = re.search(r"(рисун(ок|ке|ка))", sent, re.IGNORECASE)
    return sent[:match.end(0)] + " \\ref{%s}" % escape(img_ref) +
            sent[match.end(0):]
\end{minted}
\caption{}\label{fig:insert-ref}
\end{codewrap}

\section{Поиск и загрузка изображений}

Для загрузки изображений используется библиотека \emph{yandex-images}. Эта
библиотека использует веб-драйвер - приложение, позволяющее запускать экземпляр
браузера и осуществлять все операции с поиском и загрузкой файлов через него.

Код, используемый для загрузки изображений, приведен на рисунке
\ref{fig:download-image}.

\begin{codewrap}[0.75]
\begin{minted}[fontsize=\footnotesize]{python}
disable_dwnld = False
if not disable_dwnld:
    from yaimages.yandex_images_download.downloader
        import YandexImagesDownloader
    downloader = YandexImagesDownloader(output_directory="images/")


def download_image(context, img_file):
    if disable_dwnld:
        return
    downloader.download_images_by_keyword(context, img_file)


def close_downloader():
    if disable_dwnld:
        return
    downloader.driver.quit()
\end{minted}
\caption{}\label{fig:download-image}
\end{codewrap}

При сборке конечного документа при помощи LaTeX необходимо указать в преамбуле
путь к изображениям. Также загружаемые из Интернета приложения в большинсте
случаев будут иметь формат PNG или JPEG, поэтому необходимо использовать
соответствующие пакеты, позволяющие добавить их в документ. При этом необходимо
создать реализацию макроса \emph{\\insertimage}.

\section{Использование программмы}

Код для запуска программы приведен на рисунке \ref{fig:run-python}.

\begin{codewrap}[0.65]
\begin{minted}[fontsize=\footnotesize]{python}
with open('input.txt', encoding="utf8") as f:
    text = handle_text(f.read())
    print(text)
    with open('output.tex', encoding="utf8", mode="w") as f:
        f.write(text)
\end{minted}
\caption{}\label{fig:run-python}
\end{codewrap}

Исходный текст подается должен находиться в файле \emph{input.txt}. Для сборки
документа в формате PDF достаточно объединить команды запуска программы и сборки
в цепь: \emph{python source.py \&\& latex output.tex}.
